% <wc:start description="Content" max=800>

You get a first sense of the amount of attention paid to teaching at the
College by the fact that Harvard felt the need to launch a ``Teaching and
Learning Initiative.'' You don't expect to hear about the Red Sox launching a
``Winning the World Series Initiative'' or Charlie Sheen launching a
slightly-simpler ``Winning Initiative''---that's a core part of their
identity, not something they have to initiate.

You get a second sense by the fact that this initiative was actually launched
in 2007, and seems to have been immediately forgotten. At a poorly-attended
faculty meeting in March, 2007, the few faculty in attendance received the
``Compact on Teaching and Learning'' with disdain. A applauded remark by
Professor of Latin Katherine M. Coleman lay the blame, as always, at the feet
of distracted and disinterested students: ``Some students don’t come to
class, or they come late, or they surf the Web during lectures or even
sections, I’ve noticed.'' Obviously if someone fails to appreciate the wisdom
and clarity of your remarks, the fault must surely be theirs. But props for
noticing!

You get a third sense by observing that the College picked up this initiative
during a period of budget crisis, precisely at the moment when there are few
resources to devote to it. I should give credit where credit is due, however,
and point out that in a single year they did managed to organize
three---count them---faculty colloquia to discuss this important subject.

But if you want the follow this disinterest to the source, go to graduate
school. Graduate school is where future academics learn a lot about teaching,
most of it false.

Graduate students are taught that teaching is easy. One half-day course at
the Bok Center---all my department required---and you're ready to teach
section as a teaching fellow. Teach section for one graduate-level reading
class in a six year Ph.D. program---all my department required---and you're
ready to teach undergraduates as a junior faculty member. As a result, most
future academics (including myself) enter their first academic position not
knowing how to design a class or write a full set of lectures. More
importantly, we've given little thought to how students learn and how to use
technology effectively. We expect learning to just happen, as if by magic,
and soley due to our all-important presence.

Graduate students are taught that teaching is not important. The reason
graduate students aren't required to teach more is that faculty want it that
way: they want them in the lab doing research. Research is what earns you
your first job; research is what earns you tenure; research is what makes you
famous; research is the priority, and teaching is a distraction. Graduate
students with an interest in teaching hear warning after warning that this is
not the right thing to do. ``You've taught enough.'' ``You need more papers
on your resume, not another class.'' ``Why are you so interested in teaching?
Is your research going OK?''

Departments even establish incentives to keep students from teaching. My
department would prefer to hire undergraduates rather than distract graduate
students from their important projects. To discourage Ph.D. students
interested in education, after the mandatory one semester as a returning
``experienced'' graduate teaching fellow I was not paid for my work. Instead,
the money that the College paid the department for a graduate teaching fellow
was routed to the my adviser, to pay him for my ``lost'' research
productivity. Of course, the research papers still had to get written, and so
at the end of the day I was working harder than I would have been doing
research alone and being paid the same amount.

Finally, graduate students are taught that teaching is unrewarding.
Researched-obsessed faculty consistently define impact in ways that bring
them either extra attention, money, or---ideally---both. Few seem excited by
the chance to inspire young minds, or driven to direct their considerable
creative and analytical powers towards transforming the learning process.  My
former department has contributed in many ways to the broader Harvard
community---Professors Harry Lewis as former Dean of the College, Michael
Smith as Dean of the Faculty, Barbara Grosz as Dean of the Radcliffe
Institute, and Stuart Shieber as the Director of the Office of Scholarly
Communication---and yet no senior faculty member stepped forward to transform
CS50.

In reality, teaching isn't what graduate students are taught. Teaching is
challenging, and only becoming more so as technology changes the educational
landscape. Change has always meant that teacher and student learn somewhat
differently, but today the pace of technological advance has widened this gap
substantially. Yet, even computer scientists are still delivering lectures as
if there was no YouTube, giving PowerPoint presentations as if there was no
iPad, and designing assignments as if there was no Wikipedia.

Teaching is important, and while that alone may sound somewhat obvious,
teaching is also important at research universities. If we don't find a way
to rebalance the incentives to ensure that Harvard faculty are creative and
accomplished both as teachers and as researchers, then at some point
prospective students and their families are going to see through the myth
that just hobnobbing with Rock Star Faculty constitutes an education.

So do these mythical creatures exist, the researcher who can teach, the
teacher who can do research? I think so, because in addition to being
challenging and important, teaching can be incredibly rewarding. Not to
everyone, of course. When my former adviser Matt Welsh left Harvard for
Google he admitted that his cynical view of the impact of the academic
researcher was that the best thing that could happen to him was if a Facebook
employee read about and implemented one of his ideas. For people who define
impact in that way, academia can be a frustrating place.

But if you define impact as the opportunity to teach and inspire someone who
might go off and start Facebook, then academia can be the perfect place. (And
also, that's kind of what happened.)

% <wc:end>

\textit{Geoffrey Challen '02--'03 is a Resident Tutor at Eliot House. The
views expressed are his and do not reflect official Harvard College policy.}
